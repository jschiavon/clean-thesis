
%%%%%%%%%%%%%%%%%%%%%%%%%%%
%             PACKAGE LOADING
%%%%%%%%%%%%%%%%%%%%%%%%%%%

% Utils
    \usepackage{lipsum}
    \usepackage{etoolbox}
    \usepackage{comment}
    \usepackage{xpatch}

% Languages
    \usepackage[english]{babel}
    \usepackage{csquotes}

% Colors
    \usepackage[usenames,dvipsnames,table]{xcolor}

% Fonts
    \usepackage[no-math]{fontspec}

% Math and Algorithms
    \usepackage{amsmath,amssymb,amsfonts,amsthm}
    \usepackage{braket,cancel}
    \usepackage{xfrac,thmtools}
    \usepackage{siunitx}
    \usepackage[math-style=upright,bold-style=upright]{unicode-math-luatex}

    \usepackage[spaceRequire=false]{algpseudocodex}

% Layout, graphics and tables
    \usepackage[bottom=4.5cm]{geometry}
    \usepackage{setspace}
    \usepackage{graphicx}
    \usepackage{booktabs,caption,subcaption,multirow}
    \usepackage{tikz}

% Headers
    \usepackage[automark]{scrlayer-scrpage}

% Bibliography
    \usepackage[
        style=authoryear-comp,
        dashed=false,
        backend=biber,
%        backref=true,
        maxcitenames=2,
        maxbibnames=4,
        hyperref=true,
        uniquename=false,
        uniquelist=false,
        url=false,
        doi=false,
        isbn=false,
        date=year
    ]{biblatex}

% Acronyms
   \usepackage[printonlyused,smaller]{acronym}

% References
    \usepackage{hyperref}
    \usepackage[english,noabbrev]{cleveref}

%%%%%%%%%%%%%%%%%%%%%%%%%%%
%             MACRO DEFINITIONS
%%%%%%%%%%%%%%%%%%%%%%%%%%%

% Hypenation
    \babelhyphenation[english]{foun-ded}

% Colors
    \definecolor{UnipdRed}{RGB}{155,0,20}
    \definecolor{Trasparentino}{gray}{0.35}

% Math stuff
    \declaretheoremstyle[%
        headfont=\scshape, within=chapter,
        headpunct={\medskip}, postheadspace=\newline, notefont=\normalsize, %notebraces = {\;}{},
        spaceabove = 8pt, spacebelow = 8pt
        ]{thmstyle}
    \declaretheoremstyle[%
        headfont=\scshape, within=chapter, bodyfont=\itshape,
        headpunct={:}, postheadspace=8pt, notefont=\normalsize,
        spaceabove =8pt, spacebelow =8pt
        ]{defstyle}
    \declaretheorem[style=thmstyle]{theorem}
    \declaretheorem[style=thmstyle,sibling=theorem]{lemma}
    \declaretheorem[style=thmstyle,sibling=theorem]{corollary}
    \declaretheorem[style=defstyle]{definition}
    \declaretheorem[style=defstyle]{remark}
    \declaretheorem[style=defstyle,sibling=remark]{example}


% Algorithms
    \algrenewcommand\algorithmicrequire{\textbf{Input}}

    \algnewcommand\algorithmicbreak{\textbf{break}}
    \algnewcommand\Break{\algorithmicbreak }

% Document data
    \newcommand{\myTitle}{Thesis title}
    \newcommand{\myShortTitle}{Thesis shortitle}
    \newcommand{\mySubtitle}{Thesis subtitle}
    \newcommand{\myName}{Author Name}
    \newcommand{\myDate}{2021}
    \newcommand{\myProf}{\emph{prof}. Supervisor name}
    \newcommand{\myFaculty}{Course name}
    \newcommand{\myDepartment}{Department name}
    \newcommand{\myUni}{Università degli Studi di Padova}
    \newcommand{\myLocation}{Padova}

%%%%%%%%%%%%%%%%%%%%%%%%%%%
%             PACKAGE CONFIGURATIONS
%%%%%%%%%%%%%%%%%%%%%%%%%%%

% Fonts
    \setmainfont{TeX Gyre Pagella}[
        Numbers=OldStyle,
        Ligatures={Common,Rare,TeX}
    ]
    \newfontface\chapterNumber[Scale=2.5,Color=Trasparentino,Numbers=Lining]{TeX Gyre Pagella}

    \RequirePackage{textcase} % for \MakeTextUppercase
    \makeatletter
        \newcommand{\ct@altfont}{}
        \newcommand{\ct@caps}{\ct@altfont\scshape}

        \DeclareRobustCommand{\spacedallcaps}[1]{{\addfontfeature{LetterSpace=14.0}\ct@caps\MakeTextUppercase{#1}}}
        \DeclareRobustCommand{\spacedlowsmallcaps}[1]{{\addfontfeatures{LetterSpace=10.0}\ct@caps\MakeTextLowercase{#1}}}
        \DeclareRobustCommand{\swash}[1]{{\addfontfeature{Style=Swash}\textit{#1}}}
        \DeclareRobustCommand{\heading}[1]{{\addfontfeature{Letters=SmallCaps}#1}}
        \DeclareRobustCommand{\alternatenumberstyle}[1]{{\addfontfeature{Numbers=Lining}#1}}
    \makeatother

    \setmathfont{TeX Gyre Pagella Math}
    \setmathfont{STIX Two Math}[range={cal,bb}]
    \renewcommand{\vec}[1]{\symbf{#1}}

% Layout, graphics and tables
    \onehalfspacing
    \AfterTOCHead{\singlespacing}

    \graphicspath{{_Images/}}
    \usetikzlibrary{3d,calc}
    %\input{tikz_utils}

    \addtokomafont{caption}{\small}
    \setkomafont{captionlabel}{\scshape}

    \newcommand{\gotoclearpage}{\cleardoublepage\acresetall}

% Headers
    \clearpairofpagestyles
    \makeatletter
    \let\MakeMarkcase\spacedlowsmallcaps
    \renewcommand{\chaptermark}[1]{\markboth{\spacedlowsmallcaps{#1}}{\spacedlowsmallcaps{#1}}}
    \renewcommand{\sectionmark}[1]{\markright{\textsc{\MakeTextLowercase{\thesection}}\hspace{1em} \spacedlowsmallcaps{#1}}}
    \lehead{\mbox{\llap{\small\thepage\kern1em\color{Trasparentino}\vline}\color{Trasparentino}\hspace{0.5em}\headmark\hfil}}
    \rohead{\mbox{\hfil{\color{Trasparentino}\headmark\hspace{0.5em}}\rlap{\small{\color{Trasparentino}\vline}\kern1em\thepage}}}
    \rofoot[\mbox{\makebox[0pt][l]{\kern1em\thepage}}]{}
    \renewcommand{\headfont}{\small}
    \renewcommand{\pnumfont}{\small}
    \clearplainofpairofpagestyles
    \def\toc@heading{\chapter*{\contentsname}
        \@mkboth{\spacedlowsmallcaps{\contentsname}}{\spacedlowsmallcaps{\contentsname}}}
    \makeatother

    \cfoot[\pagemark]{}

% Chapters, sections and stuff
    \renewcommand{\descriptionlabel}[1]{\hspace\labelsep\normalfont\scshape#1}
    \setkomafont{disposition}{\rmfamily\normalfont}
    \setkomafont{chapter}{\huge\scshape}
    \setkomafont{section}{\Large\scshape}
    \setkomafont{subsection}{\large\itshape}
    \setkomafont{paragraph}{\scshape}

    \renewcommand*{\chapterformat}{\makebox[0pt][r]{\chapterNumber\thechapter\hspace{10pt}\vline\hspace{10pt}}}
    \renewcommand*{\sectionformat}{\textsc{\thesection\autodot}\enskip}
    \renewcommand*{\subsectionformat}{\textsc{\thesubsection\autodot}\enskip}
    \renewcommand*{\thechapter}{\arabic{chapter}}
    \renewcommand{\thesection}{\textsc{\roman{section}}}
    \renewcommand{\thesubsection}{\thesection.\textsc{\roman{subsection}}}
    \renewcommand{\thesubsubsection}{\thesubsection.\roman{subsubsection}}

    \makeatletter
    \renewcommand*{\p@section}{\thechapter.}
    \renewcommand*{\p@subsection}{\thechapter.}
    \renewcommand*{\p@subsubsection}{\thechapter.}
    \makeatother

    \newcommand{\cvname}[2]{%
        \begin{center}
            {\huge #1 \textsc{#2}}\\

            \medskip
            {\Large \scshape Curriculum Vitae}
        \end{center}
        \vspace*{2\baselineskip}
    }

    \newcommand{\cvsection}[1]{%
        \vspace{\baselineskip}
        \begin{flushright}
            \large \scshape #1
            \noindent\rule[0.5ex]{\linewidth}{1pt}
        \end{flushright}
    }

    \RedeclareSectionCommands[
        tocraggedpagenumber,
        toclinefill=\qquad
        ]{part}

    \RedeclareSectionCommands[
        tocraggedpagenumber,
        toclinefill=\qquad,
        afterindent=false,
        beforeskip=3.5\baselineskip,
        afterskip=2\baselineskip
        ]{chapter}

    \RedeclareSectionCommands[
        tocraggedpagenumber,
        toclinefill=\qquad,
        afterindent=false,
        beforeskip=1.5\baselineskip,
        afterskip=\baselineskip
        ]{section}

    \RedeclareSectionCommands[
        tocraggedpagenumber,
        toclinefill=\qquad,
        afterindent=false,
        beforeskip=\baselineskip,
        afterskip=0.75\baselineskip
        ]{subsection}

    \RedeclareSectionCommands[
        tocraggedpagenumber,
        toclinefill=\qquad
        ]{subsubsection}

    \DeclareNewTOC[
        type=algorithm,
        float,
        floatpos=h,
        name=Algorithm,
        listname={List of Algorithms},
        counterwithin=chapter,
        atbegin={%
            \KOMAoptions{captions=heading}
            \xpretocmd\algorithmic
            {\noindent\rule[\ht\strutbox]{\linewidth}{.8pt}\vspace{-0.8\baselineskip}}
            {}{\PatchFailed}%
            \xapptocmd\endalgorithmic
            {\noindent\rule[\ht\strutbox]{\linewidth}{1pt}\vspace{-0.8\baselineskip}}
            {}{\PatchFailed}%
        }
    ]{loalg}

    \DeclareTOCStyleEntry[
        level=1,
        indent=1.5em,
        numwidth=2.3em
    ]{default}{loalg}
    %\newcommand\listofalgorithms{\listoftoc{loalg}}

    \renewcommand*{\theequation}{%
        \ifnum\value{chapter}=0 %
            %
        \else
            \thechapter.%
        \fi
        \arabic{equation}%
    }


% Table of contents
    \setkomafont{chapterentry}{\scshape}
    \setcounter{tocdepth}{\sectiontocdepth}


% Bibliography
    \addbibresource{books.bib}
    \addbibresource{articles.bib}

    \setlength\bibitemsep{2\itemsep}
    \setlength\bibhang{2em}
    \renewcommand*{\newunitpunct}{\addcomma\space}
    \renewcommand*{\finentrypunct}{\adddot}
    \renewcommand*{\labelnamepunct}{\par}
    \renewcommand*{\intitlepunct}{\space}
    \renewcommand*{\mkbibnamefamily}[1]{\textsc{#1}}
    \renewcommand*{\mkbibnamegiven}[1]{#1}
    \renewcommand*{\mkbibnameprefix}[1]{{#1}}
    \renewcommand*{\mkbibnamesuffix}[1]{{#1}}
    \DeclareNameAlias{sortname}{family-given}
    \DeclareFieldFormat[article,inbook,incollection,inproceedings,patent,thesis,unpublished]{title}{\emph{#1\isdot}}
    \DeclareFieldFormat{journaltitle}{<<#1>>}

% References
    \hypersetup{
        pdftitle={\myTitle},          % title
        pdfauthor={\myName},   % author
        colorlinks=true,                % false: boxed links (set box color with linkbordercolor)
        breaklinks,                        % allow linebreaks on links
        linkcolor=MidnightBlue,  % color of internal links
        citecolor=MidnightBlue, %TealBlue,          % color of links to bibliography
        urlcolor=Sepia                  % color of external links
    }
    \renewcommand\UrlFont{\ttfamily}

    \newcommand{\crefrangeconjunction}{ -- }

